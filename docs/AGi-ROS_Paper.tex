\documentclass[11pt]{article}

\usepackage{times}
\usepackage{geometry}
\usepackage{hyperref}
\usepackage{graphicx}
\usepackage{amsmath}
\geometry{margin=1in}

\title{AGi-ROS: A Modular ROS 2 Framework for Bio-Inspired Robot Telemetry and Future Agentic Reasoning}
\author{OppaAI\\\texttt{update-author-info@placeholder}}
\date{\today}

\begin{document}
\maketitle

\begin{abstract}
AGi-ROS is a ROS 2 (Humble) project aimed at building a modular, bio-inspired architecture for robot systems that combine health telemetry, high-level reasoning, and extensible agent behaviors. The current release focuses on a foundational subsystem called the Vital Circulatory System (VCS), which provides heartbeat-like monitoring between robot and server processes. This paper describes the project motivation, system design, VCS implementation, and future research directions for integrating agentic planning and learning modules.
\end{abstract}

\section{Introduction}
Robotic systems require robust coordination across sensing, planning, and execution while maintaining safety and observability. AGi-ROS explores a biologically inspired framing that treats the robot-server ecosystem as a living system with vital signals, feedback loops, and modular subsystems. The current codebase emphasizes a reliable heartbeat and telemetry system as the backbone for broader agentic capabilities.

\section{System Overview}
The repository consists of two main components:
\begin{itemize}
    \item \textbf{AuRoRA} (Autonomous Rover Robot Assistant): Robot-side software that gathers hardware vitals and runs local ROS 2 nodes.
    \item \textbf{AIVA} (AI Virtual Assistant): Server-side software that receives pulses, monitors robots, and provides a coordination point.
\end{itemize}

The design is intended to evolve into a full agent loop (sense $\rightarrow$ reason $\rightarrow$ plan $\rightarrow$ act), with the VCS as the underlying circulatory layer that ensures connectivity and runtime health.

\section{Vital Circulatory System (VCS)}
The VCS is a ROS 2 subsystem that models robot-server monitoring as a heartbeat exchange. It is composed of:
\begin{itemize}
    \item \textbf{VTC (Vital Terminal Core)}: A robot-side node that collects telemetry, packages data into heartbeat pulses, and publishes them at a set rhythm.
    \item \textbf{VCC (Vital Central Core)}: A server-side node that receives pulses, estimates RTT, and returns feedback to the robot.
\end{itemize}

\subsection{Message Format}
Heartbeat messages use a custom ROS message definition called \texttt{VitalPulse}, containing identifiers, timestamps, and vital statistics such as CPU/GPU temperature. The message structure is intentionally compact to keep telemetry low-latency.

\subsection{Robot-Side Telemetry Collection}
The robot VTC uses a \textit{Pump} module to collect system health metrics through multiple flow channels (HI/MID/LO sampling rates). The pump maintains bounded buffers for time-series vitals, enabling both rapid updates and longer-term trends while staying within memory constraints.

\subsection{Server-Side Analysis}
The VCC tracks the last pulse time and flags timeouts to detect disconnections. It provides terminal-based visualization of robot identity and vital statistics to give operators immediate situational awareness.

\section{Implementation Status}
The following VCS components are implemented in the current codebase:
\begin{itemize}
    \item Robot-side VTC node with heartbeat publishing and feedback handling.
    \item Server-side VCC node with timeout handling and feedback publishing.
    \item VitalPulse message definition.
    \item Launch files for role-based deployment (robot vs. server).
\end{itemize}

Additional modules such as LLM integration, planner adapters, and simulation connectors are planned but not yet implemented in this repository snapshot.

\section{Planned Evaluation}
Future experiments will focus on:
\begin{itemize}
    \item Measuring heartbeat latency and connection robustness across network conditions.
    \item Scaling VCS to multiple robots and concurrent monitoring.
    \item Integrating VCS health signals into high-level planner decision-making.
\end{itemize}

\section{Ethics and Safety Considerations}
AGi-ROS aims to prioritize safe operation by treating health telemetry and heartbeat feedback as first-class system signals. Any integration with autonomous actuation or LLM-based reasoning should include guardrails, limits on actuation, and fallback mechanisms when telemetry indicates degraded or unsafe states.

\section{Reproducibility}
The repository provides ROS 2 packages for both robot and server components. A typical workflow is:
\begin{verbatim}
colcon build --symlink-install
source install/setup.bash
ros2 launch vcs vcs.launch.py role:=robot
ros2 launch vcs vcs.launch.py role:=server
\end{verbatim}

The project is available at \url{https://github.com/OppaAI/AGi-ROS}. Update the author contact and add citations or benchmarks before submitting to arXiv.

\section{Conclusion}
AGi-ROS provides an early-stage, bio-inspired framework for robotic systems with a focus on heartbeat-driven observability. The VCS subsystem offers a concrete, implemented foundation for future work on agentic reasoning, planning, and safe autonomy.

\end{document}
